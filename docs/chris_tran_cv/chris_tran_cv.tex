%%%%%%%%%%%%%%%%%%%%%%%%%%%%%%%%%%%%%%%%%%%%%%%%%%%%%%%%%%%%%%%%%%%%%%%%%%%%%%%%
% Medium Length Graduate Curriculum Vitae
% LaTeX Template
% Version 1.2 (3/28/15)
%
% This template has been downloaded from:
% http://www.LaTeXTemplates.com
%
% Original author:
% Rensselaer Polytechnic Institute
% (http://www.rpi.edu/dept/arc/training/latex/resumes/)
%
% Modified by:
% Daniel L Marks <xleafr@gmail.com> 3/28/2015
%
% Further modified by:
% Rohan Bavishi <rohan.bavishi95@gmail.com> 9/20/2016
%
% Important note:
% This template requires the simple_style.cls file to be in the same directory
% as the .tex file. The res.cls file provides the resume style used for
% structuring the document.
%
%%%%%%%%%%%%%%%%%%%%%%%%%%%%%%%%%%%%%%%%%%%%%%%%%%%%%%%%%%%%%%%%%%%%%%%%%%%%%%%%

%-------------------------------------------------------------------------------
%	PACKAGES AND OTHER DOCUMENT CONFIGURATIONS
%-------------------------------------------------------------------------------

%%%%%%%%%%%%%%%%%%%%%%%%%%%%%%%%%%%%%%%%%%%%%%%%%%%%%%%%%%%%%%%%%%%%%%%%%%%%%%%%
% You can have multiple style options the legal options ones are:
%
%   centered:	the name and address are centered at the top of the page
%				(default)
%
%   line:		the name is the left with a horizontal line then the address to
%				the right
%
%   overlapped:	the section titles overlap the body text (default)
%
%   margin:		the section titles are to the left of the body text
%
%   11pt:		use 11 point fonts instead of 10 point fonts
%
%   12pt:		use 12 point fonts instead of 10 point fonts
%
%%%%%%%%%%%%%%%%%%%%%%%%%%%%%%%%%%%%%%%%%%%%%%%%%%%%%%%%%%%%%%%%%%%%%%%%%%%%%%%%

\documentclass[mm]{simple_style}

% Default font is the helvetica postscript font
\usepackage{helvet}
\usepackage{hyperref}
\usepackage{url}
\usepackage{xcolor}
\hypersetup {
    colorlinks=true,
    linkcolor=colorlink,
    filecolor=magenta,
    urlcolor=colorlink,
}
\usepackage[left=0.7in, right=2in, top=0.9in]{geometry}

% Increase text height
\textheight=700pt

\begin{document}

%-------------------------------------------------------------------------------
%	NAME AND ADDRESS SECTION
%-------------------------------------------------------------------------------
%\name{Rohan Bavishi}
%\qualification{Senior Undergraduate, Computer Science, IIT Kanpur}
%\emailone{rbavishi@iitk.ac.in}
%\emailtwo{rohan.bavishi95@gmail.com}
%\website{http://home.iitk.ac.in/~rbavishi}{\url{home.iitk.ac.in/~rbavishi}}
%\github{https://github.com/rbavishi}{\url{github.com/rbavishi}}
%\phone{+91-73-180-18920}
%
%\address{C-324, Hall-1\\IIT Kanpur\\Kanpur, Uttar Pradesh, India}

\name{Christopher Tran}
\qualification{Ph.D, Computer Science, University of Illinois at Chicago}
\emailone{chris.l.tran.2016@gmail.com}
\emailtwo{ctran29@uic.edu}
\website{https://christran16.github.io/}{\url{https://christran16.github.io/}}
%\github{https://github.com/chris-tran-16}{\url{https://github.com/chris-tran-16}}
%\address{}

%-------------------------------------------------------------------------------

\begin{resume}

	%-------------------------------------------------------------------------------
	%	RESEARCH SECTION
	%-------------------------------------------------------------------------------
	\section{Professional Summary}
	\par
	I am an experienced data scientist with a strong background in research, machine learning, and causal inference. With a track record of delivering insights that drive business decisions, I have developed new algorithms machine learning and novelty detection algorithms, analyzed large datasets, and created visualizations that effectively communicate key findings to stakeholders. My experience collaborating with both internal and external teams has enabled me to mentor junior researchers and apply my expertise to a variety of domains. Looking ahead, I am excited to continue advancing my skills in data science and contribute to impactful projects that drive innovation.
	
	\vspace{-2ex}
	\sectionline
	
	%-------------------------------------------------------------------------------
	%      SKILLS
	%-------------------------------------------------------------------------------
	\section{Skills}
	
	- Expert in data analysis, data science, and machine learning using Python and statistical packages such as NumPy, Pandas, and Scikit-learn.\\
	- Familiarity with deep learning packages such as PyTorch and TensorFlow.\\
	- Proficient in data visualization using libraries like Matplotlib, Seaborn, and Plotly to communicate insights effectively.\\
	- Experience in building predictive models using techniques such as regression, decision trees, and neural networks.\\
	- Strong SQL skills for querying information from databases.
	- Experienced in Microsoft Office Suite (Word, Excel, PowerPoint).\\
	- Familiarity with the extract, transform, load (ETL) process.\\
	- Excellent written and verbal communication skills with experience in presenting findings to both technical and non-technical audiences.\\
	- Strong problem-solving abilities with attention to detail and ability to troubleshoot complex data issues.\\
	- Familiarity with other programming languages including MATLAB, Java, C++, and R, with the ability to learn new languages quickly.\\
	- Comfortable working in a fast-paced and dynamic team environment with experience in collaborating with cross-functional teams.\\
	
	\vspace{-2ex}
	\sectionline
	
	
	%\fullline
	%-------------------------------------------------------------------------------
	
	%-------------------------------------------------------------------------------
	%      UNDER REVIEW
	%-------------------------------------------------------------------------------
	
	% \section{Submissions Under Review}
	
	% \textbf{C. Tran}, E. Zheleva. ``Improving Data-driven Heterogeneous Treatment Effect Estimation Under Structure Uncertainty,'' 2021.
	
	% \vspace{-2ex}
	% \sectionline
	
	
	%-------------------------------------------------------------------------------
	%      EMPLOYMENT
	%-------------------------------------------------------------------------------
	
	
	\section{Experience}

	\begin{project}
		\title{Data Scientist}
		\supervisor{U.S.\ Securities and Exchange Commission}
		\duration{July 2022 - Present}
		\description{
			Responsibilities:
		}
		\accomplishments{
			Accomplishments:
		}
	\end{project}
	
	% \begin{minipage}{\textwidth}
	\begin{project}
		\title{Researcher}
		\supervisor{Smart Information Flow Technologies (SIFT)}
		\duration{July 2022 - Present}
		\description{
			Responsibilities:\\
			- Conduct independent and assisted research on various government contracts, including projects related to artificial intelligence, data science, and natural language processing. \\
			- Develop and implement new algorithms for detecting novelties in unknown domains. \\
			- Process, analyze, and visualize datasets to identify patterns and trends. \\
			- Apply data science and machine learning knowledge and tools to support program metrics. \\
			- Perform statistical tests and hypothesis tests to extract conclusions from data. \\
			- Collaborate with senior researchers to write reports and contracts, ensuring that all deliverables are completed on time and meet quality standards. \\
			- Mentor junior-level researchers in performing research and assigning tasks, resulting in increased productivity and improved research outcomes.
		}
		\accomplishments{
			Accomplishments: \\
			- Received several corporate recognitions for excellent research and implementation skills. \\
			- Helped achieve better novelty detection metrics, increasing scores by a factor of two \\
			- Developed tools for saving hours of work retuning detection algorithms by hand.
		}
	\end{project}
	% \end{minipage}
	
	% \begin{minipage}{\textwidth}
	\begin{project}
		\title{Research Assistant}
		% \location{University of Illinois at Chicago}
		% \supervisor{Research advisor: Dr.\ Elena Zheleva}
		\supervisor{University of Illinois at Chicago}
		\duration{Aug 2017 - May 2022}
		\description{
			Responsibilities:\\
			- Investigate and develop models to analyze heterogeneous treatment effects, accounting for individual differences and factors that influence outcome using causal trees and feature selection methods. \\
			- Conduct extensive literature reviews and stay up to date with the latest research and advancements in related fields, resulting in improved research outcomes and increased knowledge in the field. \\
			- Work with internal and external collaborators on several projects, including learning individualized differences for performance in team settings and personalized privacy. Contribute data science and machine learning knowledge to these projects, resulting in improved project outcomes and increased collaboration opportunities. \\
			- Mentored junior-level lab members on several projects, providing guidance on research methodology and contributing to their professional development.
		}
		\accomplishments{
			Accomplishments: \\
			- Co-authored over 10 papers and publications, including several that were accepted at top conferences such as KDD and AAAI. \\
			- Received research award for exceptional research promise, recognizing the impact of my research on advancing the field. \\
			- Developed research that led to an Adobe Research grant, highlighting the relevance and impact of my work in industry.\\
			- Invited to give a presentation at a conference workshop, showcasing my expertise and contribution to the field.
		}
	\end{project}
	% \end{minipage}
	
	% \begin{minipage}{\textwidth}
	\begin{project}
		\title{Research Intern}
		\supervisor{Smart Information Flow Technologies (SIFT)}
		\duration{May 2019 - Nov 2019}
		\description{
			Responsibilities:\\
			- Conduct research to model and identify factors contributing to gender bias in different countries. \\
			- Apply data science and data visualization tools to present work to internal and external collaborators. \\
			- Investigate swarm agent behaviors and develop models to better understand and predict their collective behavior pattern.\\
			- Explore and implement novel approaches for document recommendations, leveraging machine learning and social network algorithms.
		}
		\accomplishments{
			Accomplishments: \\
			- Contribute to writing final reports for contracts.\\
			- Participated in the team that received the most outstanding team award.\\
			- Received positive feedback from all senior researchers.		\\
		}
	\end{project}
	% \end{minipage}
	
	% \begin{minipage}{\textwidth}
	\begin{project}
		\title{AI Intern}
		% \location{STATS LLC}
		% \supervisor{Supervisor: Dr.\ Jennifer Hobbs}
		\supervisor{STATS Perform}
		\duration{May 2018 - Aug 2018}
		\description{
			Responsibilities:\\
			- Develop and implement models to predict ball ownership and trajectory using raw tracking data from basketball games.\\
			- Utilize deep recurrent neural networks and feature engineering to optimize model performance.\\
			- Create visualizations and reports to communicate model performance and key insights to stakeholders and team members.		\\
		}
	\end{project}
	% \end{minipage}
	
	% \begin{minipage}{\textwidth}
	\begin{project}
		\title{Teaching Assistant}
		\supervisor{University of Illinois at Chicago}
		\duration{}
		\description{
			Responsibilities:\\
			- Assist professors in hosting lab sessions for students to practice examples by hand.\\
			- Organize and grade assignments and exams.\\
			- Create new homework programs and solutions.\\
			- Teach multiple subject including machine learning, C++, MATLAB, and discrete mathematics.	\\
		}
	\end{project}
	% \end{minipage}
	
	\vspace{-2ex}
	\sectionline
	
	%-------------------------------------------------------------------------------
	%	EDUCATION SECTION
	%-------------------------------------------------------------------------------
	% \section{Education}
	% \cusemph{Indian Institute of Technology Kanpur}, Uttar Pradesh, India\\
	% {\sl Bachelor of Technology}, Computer Science and Engineering, \timeline{Jul' 13 - Jul' 17 (Expected)}\\
	% \cusemph{GPA: 9.7/10} (Overall)\\
	% \sectionline
	\section{Education}
	\cusemph{Ph.D., Computer Science}
	\\
	Advised by Dr.\ Elena Zheleva
	% \\
	% \textit{Advanced to Ph.D.\ candidacy on Oct 6, 2020}
	\\
	Department of Computer Science, University of Illinois at Chicago \timeline{Aug 2016 - Aug 2022}
	
	\cusemph{M.S., Computer Science}
	\\
	Department of Computer Science, University of Illinois at Chicago \timeline{Aug 2016 - Dec 2020}
	
	\cusemph{B.S., Computer Science}
	\\
	Department of Computer Science, Delaware State University \timeline{Aug 2012 - May 2016}
	
	\cusemph{B.S., Mathematics}
	\\
	Department of Mathematical Sciences, Delaware State University \timeline{Aug 2012 - May 2016}
	
	% \cusemph{GPA: 9.7/10} (Overall)\\
	\vspace{-2ex}
	\sectionline
	
	%-------------------------------------------------------------------------------
	%      PUBLICATIONS
	%-------------------------------------------------------------------------------
	\section{Publications}
	
	\textbf{C. Tran}, K., Burghardt, K. Lerman, E. Zheleva, ``Data-Driven Estimation of Heterogeneous Treatment Effects'' \href{https://arxiv.org/pdf/2301.06615.pdf}{[PDF]}
	
	\textbf{C. Tran}, E. Zheleva, ``Improving Data-driven Heterogeneous Treatment Effect Estimation Under Structure Uncertainty.'' \textit{To be published in ACM SIGKDD Conference on Knowledge Discovery and Data Mining, \textcolor{dorange}{KDD 2022}}. \href{https://dl.acm.org/doi/pdf/10.1145/3534678.3539444}{[PDF]}
	
	\textbf{C. Tran}, E. Zheleva, ``Heterogeneous Peer Effects in the Linear Threshold Model.'' \textit{Proceedings of the 2022 AAAI Conference on Artificial Intelligence, \textcolor{dorange}{AAAI 2022}}. \href{https://ojs.aaai.org/index.php/AAAI/article/view/20336/20095}{[PDF]}
	
	Y. He, \textbf{C. Tran}, J. Jiang, K. Burghardt, E. Ferrara, E. Zheleva, K. Lerman. ``Heterogeneous Effects of Software Patches in a Multiplayer Online Battle Arena Game,'' \textit{16th International Conference on the Foundations of Digital Games, \textcolor{dorange}{FDG 2021}}. \href{https://dl.acm.org/doi/pdf/10.1145/3472538.3472550}{[PDF]}
	
	M. T. Khan, \textbf{C. Tran}, S. Singh, D. Vasilkov, C. Kanich, B. Ur, E. Zheleva. ``Helping Users Automatically Find and Manage Sensitive, Expendable Files in Cloud Storage,'' \textit{30th USENIX Security Symposium, \textcolor{dorange}{USENIX 2021}}. \href{https://www.cs.uic.edu/~ctran/docs/khan-usenix2021.pdf}{[PDF]}.
	
	\textbf{C. Tran}, E. Zheleva. ``Heterogeneous Threshold Estimation for Linear Threshold Modeling.'' \textit{International Workshop on Mining and Learning with Graphs, \textcolor{dorange}{MLG 2020}}, Contributed Talk. \href{http://www.mlgworkshop.org/2020/papers/MLG2020_paper_23.pdf}{[PDF]}.
	
	M. Roshanaei, \textbf{C. Tran}, S. Morelli, C. Caragea, E. Zheleva. ``Paths to Empathy: Heterogeneous Effects of Reading Personal Stories Online.'' \textit{IEEE Conference on Data Science and Advanced Analytics, \textcolor{dorange}{DSAA 2019}}. \href{https://www.cs.uic.edu/~ctran/docs/roshanaei-dsaa2019.pdf}{[PDF]}.
	
	M. Mondal, G. Yilmaz, N. Hirsch, M. T. Khan, M. Tang, \textbf{C. Tran}, C. Kanich, B. Ur, E. Zheleva, ``Moving Beyond Set-It-And-Forget-It Privacy Settings on Social Media.'' \textit{26th ACM Conference on Computer and Communications Security, \textcolor{dorange}{CCS 2019}}. \href{https://www.cs.uic.edu/~ctran/docs/mondal-ccs2019.pdf}{[PDF]}.
	
	\textbf{C. Tran}, E. Zheleva, ``Learning Triggers for Heterogeneous Treatment Effects.'' \textit{Proceedings of the 2019 AAAI Conference on Artificial Intelligence, \textcolor{dorange}{AAAI 2019}}. \href{https://arxiv.org/pdf/1902.00087.pdf}{[PDF]} \href{https://github.com/edgeslab/CTL}{[Code]}.
	
	\vspace{-2ex}
	\sectionline
	
	%-------------------------------------------------------------------------------
	%      SERVICES
	%-------------------------------------------------------------------------------
	
	% \newpage
	
	\section{Service and Awards}
	
	% \cusemph{Invited talks}
	
	% - Causal Discovery Workshop at KDD (CDKDD22)
	
	\cusemph{Conference reviewer}
	
	- Causal Learning and Reasoning Conference (CLeaR) (2023)
	\\
	- AAAI Conference on Artificial Intelligence (AAAI) (2021, 2022)
	\\
	- The Web Conference (2021)
	\\
	- International Conference on Web Search and Data Mining (WSDM) (2021)
	\\
	- SIGKDD Conference on Knowledge Discovery and Data Mining (KDD) (2020)
	
	\cusemph{Awards}
	
	- UIC College of Engineering Graduate Student Award for Exceptional Research Promise (2021)
	\\
	- UIC Computer Science Graduate Student Award (2016)
	
	% \section{Talks and Presentations}
	% \sectionline
	
\end{resume}
\end{document}
