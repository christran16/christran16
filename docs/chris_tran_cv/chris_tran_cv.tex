%%%%%%%%%%%%%%%%%%%%%%%%%%%%%%%%%%%%%%%%%%%%%%%%%%%%%%%%%%%%%%%%%%%%%%%%%%%%%%%%
% Medium Length Graduate Curriculum Vitae
% LaTeX Template
% Version 1.2 (3/28/15)
%
% This template has been downloaded from:
% http://www.LaTeXTemplates.com
%
% Original author:
% Rensselaer Polytechnic Institute
% (http://www.rpi.edu/dept/arc/training/latex/resumes/)
%
% Modified by:
% Daniel L Marks <xleafr@gmail.com> 3/28/2015
%
% Further modified by:
% Rohan Bavishi <rohan.bavishi95@gmail.com> 9/20/2016
%
% Important note:
% This template requires the simple_style.cls file to be in the same directory
% as the .tex file. The res.cls file provides the resume style used for
% structuring the document.
%
%%%%%%%%%%%%%%%%%%%%%%%%%%%%%%%%%%%%%%%%%%%%%%%%%%%%%%%%%%%%%%%%%%%%%%%%%%%%%%%%

%-------------------------------------------------------------------------------
%	PACKAGES AND OTHER DOCUMENT CONFIGURATIONS
%-------------------------------------------------------------------------------

%%%%%%%%%%%%%%%%%%%%%%%%%%%%%%%%%%%%%%%%%%%%%%%%%%%%%%%%%%%%%%%%%%%%%%%%%%%%%%%%
% You can have multiple style options the legal options ones are:
%
%   centered:	the name and address are centered at the top of the page
%				(default)
%
%   line:		the name is the left with a horizontal line then the address to
%				the right
%
%   overlapped:	the section titles overlap the body text (default)
%
%   margin:		the section titles are to the left of the body text
%
%   11pt:		use 11 point fonts instead of 10 point fonts
%
%   12pt:		use 12 point fonts instead of 10 point fonts
%
%%%%%%%%%%%%%%%%%%%%%%%%%%%%%%%%%%%%%%%%%%%%%%%%%%%%%%%%%%%%%%%%%%%%%%%%%%%%%%%%

\documentclass[mm]{simple_style}

% Default font is the helvetica postscript font
\usepackage{helvet}
\usepackage{hyperref}
\usepackage{url}
\usepackage{xcolor}
\hypersetup {
    colorlinks=true,
    linkcolor=colorlink,
    filecolor=magenta,
    urlcolor=colorlink,
}
\usepackage[left=0.7in, right=2in, top=0.9in]{geometry}

% Increase text height
\textheight=700pt

\begin{document}

%-------------------------------------------------------------------------------
%	NAME AND ADDRESS SECTION
%-------------------------------------------------------------------------------
%\name{Rohan Bavishi}
%\qualification{Senior Undergraduate, Computer Science, IIT Kanpur}
%\emailone{rbavishi@iitk.ac.in}
%\emailtwo{rohan.bavishi95@gmail.com}
%\website{http://home.iitk.ac.in/~rbavishi}{\url{home.iitk.ac.in/~rbavishi}}
%\github{https://github.com/rbavishi}{\url{github.com/rbavishi}}
%\phone{+91-73-180-18920}
%
%\address{C-324, Hall-1\\IIT Kanpur\\Kanpur, Uttar Pradesh, India}

\name{Christopher Tran}
\qualification{Ph.D, Computer Science, University of Illinois at Chicago}
\emailone{chris.l.tran.2016@gmail.com}
\emailtwo{ctran29@uic.edu}
\website{https://christran16.github.io/}{\url{https://christran16.github.io/}}
%\github{https://github.com/chris-tran-16}{\url{https://github.com/chris-tran-16}}
%\address{}

%-------------------------------------------------------------------------------

\begin{resume}

%-------------------------------------------------------------------------------
%	RESEARCH SECTION
%-------------------------------------------------------------------------------
\section{Research}
\par
My research lies in the space of machine learning and causal inference, with a focus on heterogeneous treatment effect estimation: how treatment affects individuals differently. 
I am particularly interested in applications of my work in social science and personalized privacy assistants.

\vspace{-2ex}
\sectionline


%-------------------------------------------------------------------------------
%	EDUCATION SECTION
%-------------------------------------------------------------------------------
% \section{Education}
% \cusemph{Indian Institute of Technology Kanpur}, Uttar Pradesh, India\\
% {\sl Bachelor of Technology}, Computer Science and Engineering, \timeline{Jul' 13 - Jul' 17 (Expected)}\\
% \cusemph{GPA: 9.7/10} (Overall)\\
% \sectionline
\section{Education}
\cusemph{Ph.D., Computer Science}
\\
Advised by Dr.\ Elena Zheleva
% \\
% \textit{Advanced to Ph.D.\ candidacy on Oct 6, 2020}
\\
Department of Computer Science, University of Illinois at Chicago \timeline{Aug 2016 - Aug 2022}

\cusemph{M.S., Computer Science}
\\
Department of Computer Science, University of Illinois at Chicago \timeline{Aug 2016 - Dec 2020}

\cusemph{B.S., Computer Science}
\\
Department of Computer Science, Delaware State University \timeline{Aug 2012 - May 2016}

\cusemph{B.S., Mathematics}
\\
Department of Mathematical Sciences, Delaware State University \timeline{Aug 2012 - May 2016}

% \cusemph{GPA: 9.7/10} (Overall)\\
\vspace{-2ex}
\sectionline
%\fullline
%-------------------------------------------------------------------------------

%-------------------------------------------------------------------------------
%      UNDER REVIEW
%-------------------------------------------------------------------------------

% \section{Submissions Under Review}

% \textbf{C. Tran}, E. Zheleva. ``Improving Data-driven Heterogeneous Treatment Effect Estimation Under Structure Uncertainty,'' 2021.

% \vspace{-2ex}
% \sectionline

%-------------------------------------------------------------------------------
%      PUBLICATIONS
%-------------------------------------------------------------------------------
\section{Publications}

\textbf{C. Tran}, K., Burghardt, K. Lerman, E. Zheleva, ``Data-Driven Estimation of Heterogeneous Treatment Effects'' \href{https://arxiv.org/pdf/2301.06615.pdf}{[PDF]}

\textbf{C. Tran}, E. Zheleva, ``Improving Data-driven Heterogeneous Treatment Effect Estimation Under Structure Uncertainty.'' \textit{To be published in ACM SIGKDD Conference on Knowledge Discovery and Data Mining, \textcolor{dorange}{KDD 2022}}. \href{https://dl.acm.org/doi/pdf/10.1145/3534678.3539444}{[PDF]}

\textbf{C. Tran}, E. Zheleva, ``Heterogeneous Peer Effects in the Linear Threshold Model.'' \textit{Proceedings of the 2022 AAAI Conference on Artificial Intelligence, \textcolor{dorange}{AAAI 2022}}. \href{https://ojs.aaai.org/index.php/AAAI/article/view/20336/20095}{[PDF]}

Y. He, \textbf{C. Tran}, J. Jiang, K. Burghardt, E. Ferrara, E. Zheleva, K. Lerman. ``Heterogeneous Effects of Software Patches in a Multiplayer Online Battle Arena Game,'' \textit{16th International Conference on the Foundations of Digital Games, \textcolor{dorange}{FDG 2021}}. \href{https://dl.acm.org/doi/pdf/10.1145/3472538.3472550}{[PDF]}

M. T. Khan, \textbf{C. Tran}, S. Singh, D. Vasilkov, C. Kanich, B. Ur, E. Zheleva. ``Helping Users Automatically Find and Manage Sensitive, Expendable Files in Cloud Storage,'' \textit{30th USENIX Security Symposium, \textcolor{dorange}{USENIX 2021}}. \href{https://www.cs.uic.edu/~ctran/docs/khan-usenix2021.pdf}{[PDF]}.

\textbf{C. Tran}, E. Zheleva. ``Heterogeneous Threshold Estimation for Linear Threshold Modeling.'' \textit{International Workshop on Mining and Learning with Graphs, \textcolor{dorange}{MLG 2020}}, Contributed Talk. \href{http://www.mlgworkshop.org/2020/papers/MLG2020_paper_23.pdf}{[PDF]}.

M. Roshanaei, \textbf{C. Tran}, S. Morelli, C. Caragea, E. Zheleva. ``Paths to Empathy: Heterogeneous Effects of Reading Personal Stories Online.'' \textit{IEEE Conference on Data Science and Advanced Analytics, \textcolor{dorange}{DSAA 2019}}. \href{https://www.cs.uic.edu/~ctran/docs/roshanaei-dsaa2019.pdf}{[PDF]}.

M. Mondal, G. Yilmaz, N. Hirsch, M. T. Khan, M. Tang, \textbf{C. Tran}, C. Kanich, B. Ur, E. Zheleva, ``Moving Beyond Set-It-And-Forget-It Privacy Settings on Social Media.'' \textit{26th ACM Conference on Computer and Communications Security, \textcolor{dorange}{CCS 2019}}. \href{https://www.cs.uic.edu/~ctran/docs/mondal-ccs2019.pdf}{[PDF]}.

\textbf{C. Tran}, E. Zheleva, ``Learning Triggers for Heterogeneous Treatment Effects.'' \textit{Proceedings of the 2019 AAAI Conference on Artificial Intelligence, \textcolor{dorange}{AAAI 2019}}. \href{https://arxiv.org/pdf/1902.00087.pdf}{[PDF]} \href{https://github.com/edgeslab/CTL}{[Code]}.

\vspace{-2ex}
\sectionline

%-------------------------------------------------------------------------------
%      EMPLOYMENT
%-------------------------------------------------------------------------------


\section{Experience}

\begin{minipage}{\textwidth}
	\begin{project}
		\title{Researcher}
		\supervisor{Smart Information Flow Technologies (SIFT)}
		\duration{July 2022 - Present}
		\description{
			- Developed and implemented algorithms for detecting novelties in different domains \\
			- Conducted data visualization and processing to analyze data sets and identify patterns and trends \\
			- Worked on evaluating the performance of human and machine classifiers \\
			- Utilized various data science tools and techniques such as regression analysis, clustering, and classification to extract insights from data. \\
			- Conducted data cleaning and preprocessing to ensure data quality and accuracy for analysis. \\
			- Performed statistical tests and hypothesis tests to extract conclusions from data.  
		}
	\end{project}
\end{minipage}

\begin{minipage}{\textwidth}
	\begin{project}
		\title{Research Intern}
		\supervisor{Smart Information Flow Technologies (SIFT)}
		\duration{May 2019 - Nov 2019}
		\description{
		- Conducted research to model and identify factors contributing to gender bias in different countries. \\
		- Investigated swarm agent behaviors and developed models to better understand and predict their collective behavior patterns. \\
		- Explored and implemented novel approaches for document recommendations, leveraging machine learning and social network algorithms
		}
	\end{project}
\end{minipage}

\begin{minipage}{\textwidth}
	\begin{project}
		\title{AI Intern}
		% \location{STATS LLC}
		% \supervisor{Supervisor: Dr.\ Jennifer Hobbs}
		\supervisor{STATS Perform}
		\duration{May 2018 - Aug 2018}
		\description{
			- Developed and implemented models to predict ball ownership and trajectory using raw tracking data from basketball games \\
			- Utilized deep recurrent neural networks and feature engineering techniques to optimize model performance and accuracy.
		}
	\end{project}
\end{minipage}

\begin{minipage}{\textwidth}
	\begin{project}
		\title{Research Assistant}
		% \location{University of Illinois at Chicago}
		% \supervisor{Research advisor: Dr.\ Elena Zheleva}
		\supervisor{University of Illinois at Chicago}	
		\duration{Aug 2017 - May 2022}
		\description{
		- Conducted research on causal inference to better understand and identify cause-and-effect relationships in complex systems and datasets. \\
		- Investigated and developed models to analyze heterogeneous treatment effects, accounting for individual differences and factors that influence outcomes. \\
		- Explored and identified triggers for heterogeneous treatment effects - variables that maximize an estimated effect. \\
		- Conducted extensive literature reviews and stayed up-to-date with the latest research and advancements in related fields.
		}
	\end{project}
\end{minipage}

\begin{minipage}{\textwidth}
\begin{project}
	\title{Teaching Assistant}
	\supervisor{University of Illinois at Chicago}
	\duration{}
	\description{
	Introduction to Machine Learning (CS 412) \timeline{Fall 2018, Spring 2021}
	\\ 
	Instructor: Dr.\ Elena Zheleva
	\\
	- Editing and creating new additions for homework assignments, grading homework and exams, answering questions on Piazza, holding regularly scheduled office hours
	\\
	\\
	Mathematical Foundations of Computing (CS 151) \timeline{Spring 2017, Fall 2017, Spring 2018}
	\\
	Instructors: Dr.\ John Lillis, Dr.\ Gonzalo Bello
	\\ 
	- Holding multiple lab sessions per week, grading homework and exams, answering questions on Piazza, holding regularly scheduled office hours
	\\
	\\
	Program Design II (CS 141) \timeline{Summer 2017}
	\\
	Instructor: Dr.\ John Lillis
	\\
	- Grading homework and exams, assisting during lab and class sessions, holding regularly scheduled office hours
	\\
	\\
	Introduction to C/C++ with MATLAB (CS 109) \timeline{Fall 2016}
	\\
	Instructor: Dr.\ John Bell
	\\
	- Leading multiple lab sessions per week, grading homework and exams, answering questions on Piazza, holding regularly scheduled office hours
	}
\end{project}	
\end{minipage}

% \begin{project}
% 	\title{Teaching Assistant}
% 	\supervisor{University of Illinois at Chicago
% 	\\
% 	Instructors: Dr.\ John Lillis, Dr.\ Gonzalo Bello}
% 	\duration{Spring 2017, Fall 2017, Spring 2018}
% 	\description{
% 	Mathematical Foundations of Computing (CS 151) 
% 	\\
% 	- Holding multiple lab sessions per week
% 	\\
% 	- Grading homework and exams
% 	\\
% 	- Answering questions on Piazza
% 	\\
% 	- Holding regularly scheduled office hours
% 	}
% \end{project}

% \begin{project}
% 	\title{Teaching Assistant}
% 	\supervisor{University of Illinois at Chicago
% 	\\
% 	Instructor: Dr.\ John Lillis}
% 	\duration{Summer 2017}
% 	\description{
% 	Program Design II (CS 141)
% 	\\
% 	- Grading homework and exams
% 	\\
% 	- Assiting during lab and class sessions
% 	\\
% 	- Holding regularly scheduled office hours
% 	}
% \end{project}

% \begin{project}
% 	\title{Teaching Assistant}
% 	\supervisor{University of Illinois at Chicago
% 	\\
% 	Instructor: Dr.\ John Bell}
% 	\duration{Fall 2016}
% 	\description{
% 	Introduction to C/C++ with MATLAB (CS 109)
% 	\\
% 	- Holding multiple lab sessions per week
% 	\\
% 	- Grading homework and exams
% 	\\
% 	- Answering questions on Piazza
% 	\\
% 	- Holding regularly scheduled office hours
% 	}
% \end{project}

% \begin{project}
% 	\title{Teaching Assistant (Introduction to Machine Learning, CS412)}
% 	\supervisor{University of Illinois at Chicago
% 	\\
% 	Professor Elena Zheleva}
% 	\duration{F'18, Sp'21}
% 	\description{
% 		- Introduction to C/C++ with MATLAB (CS 109) \\
% 		- Program Design II (CS 141) \\
% 		- Mathematics Foundations for Computer Science (CS 151) \\
% 		- Introduction to Machine Learning (CS 412)
% 	}
% \end{project}

\vspace{-2ex}

%-------------------------------------------------------------------------------
%      SERVICES
%-------------------------------------------------------------------------------

% \newpage
\sectionline

\section{Service and Awards}

% \cusemph{Invited talks}

% - Causal Discovery Workshop at KDD (CDKDD22)

\cusemph{Conference reviewer}

- Causal Learning and Reasoning Conference (CLeaR) (2023)
\\
- AAAI Conference on Artificial Intelligence (AAAI) (2021, 2022)
\\
- The Web Conference (2021)
\\
- International Conference on Web Search and Data Mining (WSDM) (2021)
\\
- SIGKDD Conference on Knowledge Discovery and Data Mining (KDD) (2020)

\cusemph{Awards}

- UIC College of Engineering Graduate Student Award for Exceptional Research Promise (2021)
\\
- UIC Computer Science Graduate Student Award (2016)

% \section{Talks and Presentations}
% \sectionline

\end{resume}
\end{document}
